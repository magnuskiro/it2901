\section{Project Management}\label{Project Management} 
    In this section, we'll take look at how we organized the team, a brief risk assessment, and an evaluation of the work process. 
    
    \subsection{Team Organization}\label{Team Organization} 
        This section describes in detail how we organized ourselves, and how we split roles and tasks among the team members. We had a flat team structure\footnote{Flat organization structure is a structure with few or no levels of intervening management. The idea is that well-trained workers will be more productive when they are directly involved with decision making. [\url{http://en.wikipedia.org/wiki/Flat_organization}]}  and have shifted our focus accordingly over to team communication. 
    
    \subsubsection{Team Structure}\label{Team Structure}
    We already knew each other coming into the project, so we chose a flat organizational structure (Fig:~\ref{fig:TeamOrganizationchart}), with no intervening levels of management, since all decisions within the team would, to a high degree, be made by all the members together either way. Relying on the entire group for decisions both involves and invests everyone in the project and will works well with our already existing group dynamic.

    \begin{figure}[H]
        \centering
        \includegraphics[width=\textwidth]{TeamOrganizationchart}
        \caption{Team Organization chart}
        It was made during lunch, but the general principle still remains, that the structure is flat.
        \label{fig:TeamOrganizationchart}
    \end{figure}
    
    %Why changing roles does not work in our structure and how we solve the challenges that might occur in this association. 
    As the structure shows in the chart (Fig:~\ref{fig:TeamOrganizationchart}), there was no difference in what responsibility level anyone would have, or what role one had. The concept of changing roles weekly is good for a learning situation, but very inefficient where knowledge and research are key components in a limited timed project. We anticipated that the time for this project probably wouldn't be sufficient for any role changes, and therefore we had to keep people focused on the task they had been assigned to. The efficiency of the current task relied on having the current research fresh in mind. If we were to change the roles every week, the newly assigned person would spend a lot of time getting up to date at the beginning of every week, which in turn wouldn't yield any measurable gains. 
    
    Rather than focus on responsibilities within the group, we chose to focus on tasks.
The task would to some degree still represent areas of responsibilities, and since tasks would be spread across several group members, we didn't run the risk of a single missing member crippling the entire group. Instead the remaining member assigned to a task would be able to pick up the slack. This, together with thorough documentation of a members knowledge, would just about eliminate the problems associated with an absent group member.
        
    Further, the team structure and the distribution of responsibility gave us the chance to define how we want to deal with tasks and their priorities. The work flow that we had, made us prioritize tasks continuously to get the most pressing task done at the correct time. It's similar to a max heap. We put tasks in to the heap, heapify(prioritise tasks) and choose(pop) the maximized task, the task that has the highest priority. 
    
    %Task division and delegation.
    When we chose a task, we considered the person's interest, experience and existing knowledge. Most of the time, the tasks fell naturally to one person that had worked with similar tasks earlier in the project. Other times, there was more of a lottery, where the task had no prerequisites. Often we relied on a person's initiative to take a task or, we easily delegated them with a question, "Can someone do that?". Task delegation and sharing the work load has not been a problem in the project.
    
    %benefits of a flat team structure. 
    
    \subsubsection{Team communication}\label{Team communication}
    We decided that we would work together from 10 to 16, Monday through Thursday every week, with exceptions for lectures and such. Group members could also work in their free time to make up for missed collaboration hours, or to just put in some extra work. This means more work than the course requires, but we decided that we want to do it this way so that we could either take some time off now and then, or have more time for the exams in May.

    We would not be able to have frequent face to face meetings with the customer, because the customer is located in Oslo. We decided to have weekly meetings using Skype instead, as well as e-mail communication as needed. Since we have seen what happens in projects where there is little to no communication, we decided, in agreement with the costumer, that we at least wanted to have weekly meetings in order to keep a good dialog with the customer, and also give them the opportunity to take part in the development of the project. 
    
    \subsubsection{Roles}\label{roles}
    
    Our team structure is discussed in \textit{Team Structure} (ref:~\ref{Team Structure}). It describes the general structure and the ideal situation and delegation. 
    
    But to make this work in practice roles are unconsciously delegated to different people. A person ends up with a role loosely based on the first delegation of a task. 
    
    When a person works on a task, experience is gained. This experience is useful when the same, or a similar, task occurs again. 
    
    This means that the person best suited to deal with this task is the same person that dealt with it last time. To get the most efficient result the same person is picked to deal with this task as well. 

    A good thing about this is the that the same person don't have to waste time to read up on necessary information to deal with the task at hand.  
    
    How the roles actually worked is described in section ~\ref{evaluation:team organization}.
        
    \subsection{Risk Assessment}\label{Risk Assessment}
     To help us avoid most problems, we created a risk list which should contain most of the problems that we could encounter during this project. A risk list can never cover all the cases that could occur, but we think that our risk list contains most of them. To cover the last cases which we might have overlooked we decided that we would add some \textit{Risk Management} strategies to this section to explain how we will handle unforeseen risk as they appear. 
     
     To handle most of the risks that we have not written down in our risk list, we would try to have a close dialog with the customer so that if an unforeseen risk should occur, they will be informed about it. That way, we could discuss the problem together, and come up with a solution that was satisfactory for everyone. Having this open dialog with the customer also ensured that they wouldn't be surprised by any choices we made.
     
    A natural approach to risk management would be to first accept the situation and involve the whole group. Then we would assess the situation and pinpoint the problem and its cause. When the problem and it cause are found there will be a lot of different approaches to solve the situation, depending a lot on the actual situation. 
    
    When a risk is present and confirmed we have to deal with it as best we can. The way we would do this is open, frank and honest. This will put all the information on the table and we can then continue the process with a better understanding of the situation. Then we would continue with finding a solution and coordinate our collaboration within the group and with the customer and supervisor to ensure the best solution possible.

    \subsection{Progress tracking and Documentation}\label{Progress tracking and Documentation}
    In the beginning we had a summary every day where we wrote what we were working on and what had to be done. We stopped doing this after we got good activity plans because the daily summaries became unnecessary. 
    
    \begin{figure}[H]
        \centering
        \includegraphics[width=\textwidth]{Week8activityplan}
        \caption{Example Activity plan}
        This figure displays the structure of our activity plans. It's meant as an overview. See the appendix for full weekly reports. 
        (ref:~\ref{Attachments:Activity Plans})
        \label{fig:Week8activityplan}
    \end{figure}
    
    The activity plans(Fig:~\ref{fig:Week8activityplan}) now have the role of our day to day summaries and work progress. We updated the activity plan as we went along. This way we had a complete overview of tasks and work hours that we were planning that week. As we updated the activity plan, we had an overview of the work done that week and where we had missed with our time estimation. 
    
    \begin{figure}[H]
        \centering
        \includegraphics[scale=0.5]{Week7statusreport}
        \caption{Status report example}
        This is an example of one of our status reports. We create them every week. All the weekly reports can be found in appendix ~\ref{Attachments:Weekly Reports}
        \label{fig:Week7statusreport}
    \end{figure}
    
    As can be seen in the weekly report (Fig:~\ref{fig:Week7statusreport}), the status report had a standard setup. We created a template early in the project, so that we could reuse it later and reduce work. In the process of creating the template, we put some thought into it so that we would get a template that would work throughout the project without further changes. Despite the thought process of creating the templates, we had to make some small changes throughout the project.
    
       \subsection{Progress Evaluation}\label{Progress Evaluation}
    We assessed our progress every week. This has been done with weekly reports, schedules and activity plans. These can be seen in the attachments appendix ~\ref{Attachments}.
    
    This documentation process took a substantial amount of time every Thursday. The documentation part that we did every week, consisted mainly of three things. Writing the weekly report, updating the activity plan and creating the next activity plan. 
    
    The time spent on this varied quite a bit. This was mainly due to a variation in how accurate our plans were, compared to what we managed to do that week. Sometimes everything went according to the plan, and there was little to write about in the report. Other times, me miscalculated greatly, and had to transfer tasks to the following week or find new ones, because we had finished all of the tasks on the plan.
    
    The weekly reports and activity plans have been used to inform the customer and supervisor, and get feedback on our progress.
    
    The schedules tracked our time usage and is kind of a log for what has been done by whom. We did not use the schedules to point out people that had put down more or less work in this project. It is merely a document for the group to keep track of things. 
    
    
